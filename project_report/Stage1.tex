\textnormal{
. Users interested in entities like organizations, companies, people etc. can use it as a single source where all other data of interest is available. Enterprises can use the time line view to analyse their past events and how they helped in there advancement. Our system will also be exposing a REST api for researchers interested in our event data. It can be used to map the events with the stock prices or marketing decisions, which can help them come up with better approaches for the same.

Get a realistic project idea that includes potential real world scenarios,
with a description of the different user types along with their interactions with the system 
as well as the system feedback to them, according to their information needs. 
This stage also requires the specification of the different constraints and restrictions
that need to be enforced depending on the different types of user (system interactions). 
The deliverables for this stage include the following items:
} 

\begin{itemize} 
\item{Our project is like a minimalistic search engine which gives a collated information in form of a time line.} 

\item{ This system can be used by people exploring the web for information, it can be used by enterprises and also by researchers analysing statistical data. }
	
\item{ Enterprises can use the time line view to analyse their past events and how they helped in there advancement. They can also use it to analyse their competitor's timeline to study and compare the expansion over the time.}

\item{ Our system will also be exposing a REST api for researchers interested in our event data. Researchers can use this event data and map it with other statistical information like variation of stock prices over the time, popularity index etc. to get a newer perspective on data.}

\end{itemize}

Please insert your deliverables for Stage1 as follows:

\begin{itemize} 
\item{The general system description: } 
Our project is like a minimalistic search engine which gives collated information about the searched item in form of a time line.
\item{The three types of users (grouped by their data access/update rights): }
People exploring web, Enterprises and researchers interested in event data, as follows:
\item{The user's interaction modes: }
Please insert the user's interaction modes here.
\item{The real world scenarios: }
Please insert the real world scenarios in here, as follows.
	\begin{itemize} 
	\item{Scenario1 description: }
	Please insert Scenario1 description here.
	\item{System Data Input for Scenario1: }
	Please insert System Data Input for Scenario1 here.
	\item{Input Data Types for Scenario1: }
	Please insert Input Data Types for Scenario1 here.
	\item{System Data Output for Scenario1: }
	Please insert System Data Output for Scenario1 here.
	\item{Output Data Types for Scenario1: }
	Please insert Output Data Types for Scenario1 in here.
	\item {Please repeat that pattern for each scenario (at least 2 scenarios per user).}
	\end{itemize}
Please repeat that pattern for each user type.
\item{ Project Time line and Divison of Labor.}
The project has three main components - Web-scrapper over hadoop, Web-application with mongoDB backend and Machine learning techniques used over hadoop to extract events of interest. Each of the team member will be responsible for completion of one of the above tasks, including work related to development, testing and documentation. We estimate the project completion in about 6 weeks. 

By the end of week 1 we expect to have identified all the seeding sources for our application and built a working web-scraper with a hadoop backend running in pseudo-distributed mode. After we have tested it's working on localhost, 1 person in team will be responsible for deploying hadoop in fully-distributed mode and running the web-scrapper on top of it. In parallel other members will start building the web application and exploring the machine learning techniques. At the end of week 2 we expect to have a web-scraper running on top of a fully-distributed hadoop and a functional web application front end. In week 3 we will link our web application with a mongoDB backend which will be loaded with some dummy data to make the web application fully operational to test its functionality. In parallel we will also be working on cleaning the data and exploring machine learning techniques to extract event data from it. At the end of week 4 we should be able to extract useful event information from hadoop and pass it onto mongoDB to be available for query from web application. In week 5 we will work on making the application stable. At the end of week 6 we should have a stable working system with final project report and presentation.
\end{itemize}
}
