\documentclass[10pt,conference]{IEEEtran}

\ifCLASSINFOpdf
	\usepackage[pdftex]{graphicx}
	\graphicspath{{./figures/}}
	\DeclareGraphicsExtensions{.pdf,.jpeg,.png}
\else
	\usepackage[dvips]{graphicx}
	%\graphicspath{{./figures/}}
	\DeclareGraphicsExtensions{.eps}
\fi

\usepackage[cmex10]{amsmath}
\usepackage[tight,footnotesize]{subfigure}
\usepackage{xcolor}
\usepackage[lined,ruled]{algorithm2e}
\usepackage[latin1]{inputenc}
\usepackage{tikz}
\usetikzlibrary{shapes}
\usetikzlibrary{arrows}

\newtheorem{property}{Property}
\newtheorem{proposition}{Proposition}
\newtheorem{theorem}{Theorem}
\newtheorem{conjecture}{Conjecture}
\newtheorem{question}{Question}
\newtheorem{definition}{Definition}
\newtheorem{corollary}{Corollary}

\makeatletter
\pgfdeclareshape{datastore}{%
\inheritsavedanchors[from=rectangle]
\inheritanchorborder[from=rectangle]
\inheritanchor[from=rectangle]{center}
\inheritanchor[from=rectangle]{base}
\inheritanchor[from=rectangle]{north}
\inheritanchor[from=rectangle]{north east}
\inheritanchor[from=rectangle]{east}
\inheritanchor[from=rectangle]{south east}
\inheritanchor[from=rectangle]{south}
\inheritanchor[from=rectangle]{south west}
\inheritanchor[from=rectangle]{west}
\inheritanchor[from=rectangle]{north west}
\backgroundpath{%
  %  Store lower right in xa/ya and upper right in xb/yb
  \southwest \pgf@xa=\pgf@x \pgf@ya=\pgf@y
  \northeast \pgf@xb=\pgf@x \pgf@yb=\pgf@y
  \pgfpathmoveto{\pgfpoint{\pgf@xa}{\pgf@ya}}
  \pgfpathlineto{\pgfpoint{\pgf@xb}{\pgf@ya}}
  \pgfpathmoveto{\pgfpoint{\pgf@xa}{\pgf@yb}}
  \pgfpathlineto{\pgfpoint{\pgf@xb}{\pgf@yb}}
  }
}
\makeatother

\newcommand{\riham}[1]{{\color{red}{#1}}}
\newcommand{\james}[1]{{\color{blue}{#1}}}
\newcommand{\yx}[1]{{\color{red}{#1}}}


\begin{document}

\title{CS512 (Fall 2015) - Project Playback}

\author{%
\IEEEauthorblockN{Ashish Jindal}
\IEEEauthorblockA{Rutgers University\\
Piscataway, NJ, USA\\
Email: ashish.jindal@rutgers.edu}
\and
\IEEEauthorblockN{Yikun Xian}
\IEEEauthorblockA{Rutgers University\\
Piscataway, NJ, USA\\
Email: siriusxyk@gmail.com}
\and
\IEEEauthorblockN{Sanjivi Muttena}
\IEEEauthorblockA{Rutgers University\\
Piscataway, NJ, USA\\
Email: sm1727@scarletmail.rutgers.edu}
}

\maketitle

\begin{abstract}
\textnormal{%
Events in the past play a major role in shaping our future. World Wide Web has
abundant information related to such events, but it is scattered all over the
web. There is no automated and efficient way to organize this information and
to study such events for a specific entity of interest, where entity here
refers to a generic term to represent an organization, country and person etc.
This project aims to build a web-based automate system that can collect such
data from heterogeneous online sources, extract condensed description of
events, and integrate them into a timeline pertaining to an entity. This
largely benefits those who would like to study history events of a certain
company or organization in chronological order. 
}
\end{abstract}

\begin{IEEEkeywords}
  \textnormal{Heterogeneous Sources, Event Extraction, Topic Model, Text Mining}
\end{IEEEkeywords}

\IEEEpeerreviewmaketitle

\section{Project Description}\label{sec:project-description}

Project Playback is about gathering data from web using various web scrapping
methods and then extracting information about events of interest from the
collected data using several machine learning algorithms. We will use Java
to develop most of our system components along with a backend built using
Hadoop and MongoDB. Our project falls in two categories \textit{Massive
Algorithmics} and \textit{Scalable Algorithms Infrastructure}.

We believe that a timeline view of events and activities for an entity of
interest will give a new perspective for exploring data on Internet. Instead of
viewing information organized in form of sections we will be showing it
organized in form of a timeline. The intent is to give a bird's eye view of
evolution of the entity over the time and study that evolution process based on
data we call as the events of interest.

It is feasible to complete the system in 6 weeks if we consider only a limited
set of entities to analyse. For example, if we limit the system's input to only
consider fortune 500 companies or 200 most significant people in computer
science field, then we should be able to deliver a working system within 6
weeks. The main stumbling blocks for this project are identifying seeding
sources for data, integrating a real-time web-crawler into the system, cleaning
the retrieved data, extracting events of condensed description, reorganizing
events of companies and finally visualizing timeline data in a web-based
system.

The project has four stages: requirement gathering, design, infrastructure
implementation, and user interface.

\subsection{Stage1 - The Requirement Gathering Stage}
\label{sec:requirement-gathering-stage}
\subsection{Stage 1 - The Requirement Gathering Stage}
\label{sec:stage1}

Following are the deliverables for this stage:

\paragraph{General Description}
This project is about setting up an infrastructure for big data analytics using
Hadoop and demonstrate a data analytics application using event data from
various sources like news and blogs, etc. The first part is to construct a
distributed computing platform based on Hadoop and its high-level applications
like Mahout. The system provides some well-encapsulated APIs for training and
testing general models. The second part include a real application on event
analysis driven by Hadoop. It mainly demonstrate the feasibility of
architecture to apply Hadoop into existing web-based application.

\paragraph{User Type 1}

People interested in analyzing data that can't be stored on a single machine.

\begin{itemize}
\item User Interaction Modes: Hive query language.
\item Real World Scenarios:
  \begin{itemize}
  \item Scenario 1 Description: Processing of large server logs.
  \item System Data Input for Scenario 1: Log files.
  \item Input Data Types for Scenario 1: Structured log files.
  \item System Data Output for Scenario 1: Information based on query.
  \item Scenario 2 Description: Text mining.
  \item System Data Input for Scenario 2: Big sources of textual information.
  \item Input Data Types for Scenario 2: Text.
  \item System Data Output for Scenario 2: Information based on query.
  \end{itemize}
\end{itemize}

\paragraph{Project Timeline and Division of Labor}

We will start with studying about Hadoop and map reduce systems and then  set
up a Hadoop system on local host. When we are successful in doing that we will
find a small dataset, large enough to fit on a single system and start test
analysis over it using Hadoop. Then we will repeat the same process by setting
up Hadoop in pseudo distributed mode. Parallely two team members will start
building the demo analytics application over Hadoop. In the end we will setup
Hadoop on multiple clusters and deploy our application over it.

\begin{itemize}
  \item Week 1 and Week 2: Work on requirement gathering and design of system
    using Hadoop. Setup Hadoop on local host as a single cluster and perform
    dummy analysis to test its functionality.
  \item Week 3 and week 4: Setup Hadoop infrastructure over multiple clusters
    and create an analytics application demonstrating big data analysis using
    Hadoop.
  \item Week 5: Test the application and prepare the necessary documentation \&
    project report.
\end{itemize}


%\subsection{Stage2 - The Design Stage. }\label{sec: 2:The Design Stage.}
%%%%%%%%%%%%%%%%%%%%%%%%%%%%%%%%%%%%%%%%%%%%%%%%%%%%%%%%%%%%%%%%%%%%%%%%%%%%%%%%%%%%%%%%%%%%%%%%%%%%%%%%%%
%\input{Stage2.tex}

%\subsection{Stage3 - The Implementation Stage. }\label{sec: 3 The Implementation Stage.}
%%%%%%%%%%%%%%%%%%%%%%%%%%%%%%%%%%%%%%%%%%%%%%%%%%%%%%%%%%%%%%%%%%%%%%%%%%%%%%%%%%%%%%%%%
%\input{Stage3.tex}

%\subsection{Stage4 -	User Interface. }\label{sec: 4. User Interface.}
%%%%%%%%%%%%%%%%%%%%%%%%%%%%%%%%%%%%%%%%%%%%%%%%%%%%%%%%%%%%%%%%%%%%%%%%%%%%%%%%%%%%%%%%%%%%%%%%%%%%%%%%%%
%\input{Stage4.tex}

%\section{Project Highlights.}\label{sec:7. Project Highlights.}
%%%%%%%%%%%%%%%%%%%%%%%%%%%%%%%%%%%%%%%%%%%%%%%%%%%%%%%%%%%%%%%%%%%%%%%%%%%%%%%%%%%%%%%%%%%%%%%%%%%%%%%%%%
%\input{Highlights.tex}


\bibliographystyle{IEEEtran}
\bibliography{main}
\nocite{*}

\end{document}
