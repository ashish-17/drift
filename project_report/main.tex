\documentclass[10pt,conference]{IEEEtran}

\ifCLASSINFOpdf
	\usepackage[pdftex]{graphicx}
	\graphicspath{{./figures/}}
	\DeclareGraphicsExtensions{.pdf,.jpeg,.png}
\else
	\usepackage[dvips]{graphicx}
	%\graphicspath{{./figures/}}
	\DeclareGraphicsExtensions{.eps}
\fi

\usepackage[cmex10]{amsmath}
\usepackage[tight,footnotesize]{subfigure}
\usepackage{xcolor}
\usepackage[lined,ruled]{algorithm2e}
\usepackage[latin1]{inputenc}
\usepackage{tikz}
\usetikzlibrary{shapes}
\usetikzlibrary{arrows}

\newtheorem{property}{Property}
\newtheorem{proposition}{Proposition}
\newtheorem{theorem}{Theorem}
\newtheorem{conjecture}{Conjecture}
\newtheorem{question}{Question}
\newtheorem{definition}{Definition}
\newtheorem{corollary}{Corollary}

\makeatletter
\pgfdeclareshape{datastore}{%
\inheritsavedanchors[from=rectangle]
\inheritanchorborder[from=rectangle]
\inheritanchor[from=rectangle]{center}
\inheritanchor[from=rectangle]{base}
\inheritanchor[from=rectangle]{north}
\inheritanchor[from=rectangle]{north east}
\inheritanchor[from=rectangle]{east}
\inheritanchor[from=rectangle]{south east}
\inheritanchor[from=rectangle]{south}
\inheritanchor[from=rectangle]{south west}
\inheritanchor[from=rectangle]{west}
\inheritanchor[from=rectangle]{north west}
\backgroundpath{%
  %  Store lower right in xa/ya and upper right in xb/yb
  \southwest \pgf@xa=\pgf@x \pgf@ya=\pgf@y
  \northeast \pgf@xb=\pgf@x \pgf@yb=\pgf@y
  \pgfpathmoveto{\pgfpoint{\pgf@xa}{\pgf@ya}}
  \pgfpathlineto{\pgfpoint{\pgf@xb}{\pgf@ya}}
  \pgfpathmoveto{\pgfpoint{\pgf@xa}{\pgf@yb}}
  \pgfpathlineto{\pgfpoint{\pgf@xb}{\pgf@yb}}
  }
}
\makeatother

\newcommand{\riham}[1]{{\color{red}{#1}}}
\newcommand{\james}[1]{{\color{blue}{#1}}}
\newcommand{\yx}[1]{{\color{red}{#1}}}


\begin{document}

\title{A Hadoop-driven Data Analysis System}

\author{%
\IEEEauthorblockN{Ashish Jindal}
\IEEEauthorblockA{Rutgers University\\
Piscataway, NJ, USA\\
Email: ashish.jindal@rutgers.edu}
\and
\IEEEauthorblockN{Yikun Xian}
\IEEEauthorblockA{Rutgers University\\
Piscataway, NJ, USA\\
Email: siriusxyk@gmail.com}
\and
\IEEEauthorblockN{Sanjivi Muttena}
\IEEEauthorblockA{Rutgers University\\
Piscataway, NJ, USA\\
Email: sm1727@scarletmail.rutgers.edu}
}

\maketitle

\begin{abstract}
\textnormal{%
The complexity of modern analytics needs is outstripping the available
computing power of legacy systems. Distributed system like Hadoop compliments
this requirement for storing and analyzing huge sets of information by
providing a platform for parallel processing of large data sets stored over
multiple machines. This project aims to setup a Hadoop infrastructure and
demonstrate its power for big data analysis. Some machine learning models will
also be deployed in this system so that developers can directly call
corresponding APIs to execute training and testing models.
}
\end{abstract}

\begin{IEEEkeywords}
  \textnormal{Hadoop-driven, Data Analysis, Infrastructure}
\end{IEEEkeywords}

\IEEEpeerreviewmaketitle

\section{Project Description}\label{sec:project-description}

Apache Hadoop is an open-source software framework written in Java for
distributed storage and distributed processing of very large data sets on
computer clusters built from commodity hardware. Hadoop stores data as it comes
in - structured or unstructured - saving time on configuring data for
relational databases. In our project we will store a large dataset in Hadoop
file system and use it for analysis of events from various sources like news,
twitter etc. Our project falls in the category \textit{Scalable Algorithms
Infrastructure}. The main stumbling points for this project are identifying a
large dataset for Hadoop, configuring Hadoop for multiple clusters and building
a data analytics system over Hadoop.

The project has four stages: requirement gathering, design, infrastructure
implementation, and user interface.

\subsection{Stage1 - The Requirement Gathering Stage}
\label{sec:requirement-gathering-stage}
\subsection{Stage 1 - The Requirement Gathering Stage}
\label{sec:stage1}

Following are the deliverables for this stage:

\paragraph{General Description}
This project is about setting up an infrastructure for big data analytics using
Hadoop and demonstrate a data analytics application using event data from
various sources like news and blogs, etc. The first part is to construct a
distributed computing platform based on Hadoop and its high-level applications
like Mahout. The system provides some well-encapsulated APIs for training and
testing general models. The second part include a real application on event
analysis driven by Hadoop. It mainly demonstrate the feasibility of
architecture to apply Hadoop into existing web-based application.

\paragraph{User Type 1}

People interested in analyzing data that can't be stored on a single machine.

\begin{itemize}
\item User Interaction Modes: Hive query language.
\item Real World Scenarios:
  \begin{itemize}
  \item Scenario 1 Description: Processing of large server logs.
  \item System Data Input for Scenario 1: Log files.
  \item Input Data Types for Scenario 1: Structured log files.
  \item System Data Output for Scenario 1: Information based on query.
  \item Scenario 2 Description: Text mining.
  \item System Data Input for Scenario 2: Big sources of textual information.
  \item Input Data Types for Scenario 2: Text.
  \item System Data Output for Scenario 2: Information based on query.
  \end{itemize}
\end{itemize}

\paragraph{Project Timeline and Division of Labor}

We will start with studying about Hadoop and map reduce systems and then  set
up a Hadoop system on local host. When we are successful in doing that we will
find a small dataset, large enough to fit on a single system and start test
analysis over it using Hadoop. Then we will repeat the same process by setting
up Hadoop in pseudo distributed mode. Parallely two team members will start
building the demo analytics application over Hadoop. In the end we will setup
Hadoop on multiple clusters and deploy our application over it.

\begin{itemize}
  \item Week 1 and Week 2: Work on requirement gathering and design of system
    using Hadoop. Setup Hadoop on local host as a single cluster and perform
    dummy analysis to test its functionality.
  \item Week 3 and week 4: Setup Hadoop infrastructure over multiple clusters
    and create an analytics application demonstrating big data analysis using
    Hadoop.
  \item Week 5: Test the application and prepare the necessary documentation \&
    project report.
\end{itemize}


%\subsection{Stage2 - The Design Stage. }\label{sec: 2:The Design Stage.}
%%%%%%%%%%%%%%%%%%%%%%%%%%%%%%%%%%%%%%%%%%%%%%%%%%%%%%%%%%%%%%%%%%%%%%%%%%%%%%%%%%%%%%%%%%%%%%%%%%%%%%%%%%
%\input{Stage2.tex}

%\subsection{Stage3 - The Implementation Stage. }\label{sec: 3 The Implementation Stage.}
%%%%%%%%%%%%%%%%%%%%%%%%%%%%%%%%%%%%%%%%%%%%%%%%%%%%%%%%%%%%%%%%%%%%%%%%%%%%%%%%%%%%%%%%%
%\input{Stage3.tex}

%\subsection{Stage4 -	User Interface. }\label{sec: 4. User Interface.}
%%%%%%%%%%%%%%%%%%%%%%%%%%%%%%%%%%%%%%%%%%%%%%%%%%%%%%%%%%%%%%%%%%%%%%%%%%%%%%%%%%%%%%%%%%%%%%%%%%%%%%%%%%
%\input{Stage4.tex}

%\section{Project Highlights.}\label{sec:7. Project Highlights.}
%%%%%%%%%%%%%%%%%%%%%%%%%%%%%%%%%%%%%%%%%%%%%%%%%%%%%%%%%%%%%%%%%%%%%%%%%%%%%%%%%%%%%%%%%%%%%%%%%%%%%%%%%%
%\input{Highlights.tex}


\bibliographystyle{IEEEtran}
\bibliography{main}
\nocite{*}

\end{document}
